\documentclass{article}
\usepackage{graphicx} % Required for inserting images

\title{TFM}
\author{Enric Bassó Xalabardé}
\date{November 2023}

\begin{document}

\maketitle
\tableofcontents

\section{Explicació detallada del problema a resoldre}
Possibles temes:
\subsection{Selecció i optimització de dades de prova impulsada per AI per a simulacions HPC}
Simulacions HPC poden generar grans quantitats de dades. Pot fer difícil la selecció i optimització de dades de prova. L'IA es pot utilitzar per automatitzar aquest procés i assegurar-se que les dades de prova siguin representatives de les condicions del món real que la simulació pretén modelar.

\subsection{Anàlisi de resultats de proves i detecció de errors basada en AI per a aplicacions de processament de grans quantitats de dades}
Pot ser Difícil identificar la causa arrel dels errors en big data. L'IA es pot utilitzar per analitzar els resultats de les proves i identificar patrons que indiquin la presència d'errors. Això pot ajudar els desenvolupadors a identificar i corregir errors de manera més ràpida i eficient.

\subsection{Generació de casos de proves per AI per a simulacions basades en física}
La creació de casos de proves per a simulacions fisiques pot ser laboriosa i complicada. L'IA es pot utilitzar per generar casos de proves automàticament, basant-se en les lleis físiques que regeixen el sistema.

\subsection{Automatització de la prova de càlculs complexos amb AI:}
Càlculs matemàtics i exploració de com automatitzar les proves per a aplicacions que involucren matemàtiques complexes.

\subsection{Paral·lelització eficient en l'automatització de proves amb AI:}
optimització de l'eficiència de l'automatització de proves mitjançant la paral·lelització.
algoritmes d'IA per a una assignació dinàmica de recursos


\subsection{Paral·lelització eficient en l'automatització de proves:}
Utilitza la computació d'alt rendiment per optimitzar l'automatització de proves mitjançant la paral·lelització.
Investiga mètodes per distribuir casos de prova eficientment.

\subsection{Proves impulsades per dades per a aplicacions de Big Data:}
Disseny de proves automatitzades per a aplicacions que manipulen conjunts de dades grans.
Investiga com les proves impulsades per dades poden millorar la validació d'aplicacions intensives en dades.

\subsection{Proves de rendiment per a HPC:}
Desenvolupar estratègies de proves de rendiment.
Investiga tècniques per avaluar i optimitzar el rendiment d'aplicacions en un entorn de computació d'alt rendiment.

\subsection{Integració de la Simulació en l'Automatització de Proves:}
Combina la teva experiència en simulació amb l'automatització de proves per crear un marc de treball per a proves d'aplicacions a través d'entorns simulats.
Investiga els avantatges de l'ús de simulacions en el procés de proves.

\subsection{Altres possibilitats}
\begin{itemize}
\item Millora del rendiment de l'automatització de proves per a aplicacions de processament de grans quantitats de dades.
\item Desenvolupament d'un marc de treball d'automatització de proves per a simulacions basades en física.
\item Ús d'aprenentatge automàtic per millorar l'exactitud i l'eficiència de l'anàlisi de resultats de proves.
\item Millorar de l'escalabilitat de l'automatització de proves per a sistemes de programari grans i complexos.
\item Desenvolupament d'un marc de treball d'automatització de proves pde continuous integration (CI/CD).
\end{itemize}
\section{Enumeració dels objectius que es volen aconseguir amb la realització del TFM.}
\section{escripció de la metodologia que es seguirà durant el desenvolupament del TFM.}
\section{Llista de les tasques a realitzar per aconseguir els objectius descrits.}
\section{Planificació temporal detallada d'aquestes tasques i les seves dependències.}
\section{Iniciar la revisió de l'estat de l'art o estudi de mercat.}
\section{Estudi de l'impacte ètic, social i ambiental.} 

\end{document}
